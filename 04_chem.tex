\section{化学动力学}
\subsection{概述}
\subsection{总包反应与基元反应}
燃料和氧化剂的\textbf{总包反应机理}可以被写作:
\begin{equation}
    \mathrm{F} + a\mathrm{Ox}\leftarrow b\mathrm{Pr}
\end{equation}

反应速率可以被表达为:

\begin{equation}
    {\frac{\mathrm{d}[X_{F}]}{\mathrm{d}t}}=-k_{G}(T)[X_{F}]^{n}[X_{O x}]^{m},
\end{equation}其中\(k_G\)为\textbf{总包反应速率常数},\(n\), \(m\)为\textbf{反应级数}。这个式子只在特定的温度和压力范围适用,并且与用于确定反应速率参数的实验装置有关。为了描述一个总体反应所需要的一组基元反应称为\textbf{反应机理}.

\textbf{基团}或\textbf{自由基}是指具有反应性的分子或原子,拥有不成对的电子。

\subsection{基元反应速率}
\subsubsection{双分子反应和碰撞理论}
大部分的基元反应是\textbf{双分子反应}:

\begin{equation}
    \mathrm{A}+\mathrm{B}\rightarrow \mathrm{C} + \mathrm{D}
\end{equation}

反应速率可以写作:
\begin{equation}
    {\frac{\mathrm{d}[{A}]}{\mathrm{d}t}}=-k_{\mathrm{bimolec}}[{A}][{B}].
\end{equation}
如果研究问题的温度范围不是很大,双分子反应速率常数可以用经验的阿累尼乌斯形式(Arrheniusform)来表示,即:
\begin{equation}
    k(T) = A\exp(-E_A/R_u T)
\end{equation}这里的\(A\)是\textbf{指前因子}或\textbf{频率因子}。严格来说它与\(T^{1/2}\)相关。

也有写作:
\begin{equation}
    k(T)=A T^{b}\exp(-E_{A}/R_{u}T),
\end{equation}

\subsubsection{其他基元反应}

\textbf{单分子反应:}
\begin{equation}
    \mathrm{A\rightarrow B}
\end{equation}
或者:
\begin{equation}
    \mathrm{A}\rightarrow \mathrm{B} + \mathrm{C}
\end{equation}

在高压的情况下,这个反应是一阶的,反应速率为:
\begin{equation}
    \frac{\dd \mathrm{A}}{\dd t}=-k_{\mathrm{uni}}[\mathrm{A}]
\end{equation}
在低压时,它还与任意分子的浓度有关,
\begin{equation}
    \frac{\dd[\mathrm{A}]}{\dd t}=-k[\mathrm{A}][\mathrm{M}]
\end{equation}这里的M是任意分子。

\textbf{三分子反应}
\begin{equation}
    \mathrm{A + B + M \rightarrow C + M}
\end{equation}

这里的M同样是任意分子。

\textbf{第三体的作用}:在自由基-自由基反应中,第三体的作用是携带走在形成稳定的组分时释放出来的能量。在碰撞的过程中,新形成的分子的内能传递给第三体 M,成为 M 的动能。没有这一能量的传递,新形成的分子将重新离解为组成它的原子。

\subsection{多步反应机理的反应速率}
\subsubsection{净生成率}
\subsubsection{净生成率的简洁表达式}
对于反应机理,表达式可以写为
\begin{equation}
    \sum_{j=1}^{N}\nu_{j}^{\prime}\,X_{j}\Leftrightarrow\sum_{j=1}^{N}\nu_{j i}^{\prime\prime}\,X_{j}\quad{\mathrm{for}}\quad i=1,2,\dots,L\,,
\end{equation}
净生成率被写作:
\begin{equation}
    \dot{\omega_j}=\sum_{i=1}^L\nu_{ji} q_i\quad{\mathrm{for}}\quad j=1,2,\dots,N\,,
\end{equation}
\begin{equation}
    \nu_{ji} = (\nu_{ji}'' - \nu_{ji}')
\end{equation}
\begin{equation}
    q_i = k_{fi}\Pi_{j=1}^N[X_j]^{\nu_{ji}'} - k_{ri}\Pi_{j=1}^N[X_j]^{\nu_{ji}''}
\end{equation}

\subsubsection{反应速率常数与平衡常数之间的关系}
速率常数不好测,基于热力学测量与计算的平衡常数好测。

\begin{equation}
    \frac{k_f(T)}{k_r(T)} = K_c(T)
\end{equation}

这里的\(K_c\)是基于浓度的平衡常数:
\[K_p = K_c (R_u T/ P^0)^{c+d+\cdots-a-b-\cdots}=K_c (R_u T/ P^0)^{\sum \nu''-\sum \nu'}\]

实际操作中,测定好测的正反应速率常数,然后推算出逆反应的速率常数。

\subsubsection{稳态近似}
在燃烧过程所涉及的许多化学反应系统中,会形成许多高反应性的中间产物,即自由基。针对这类中间产物或自由基,采用稳态近似,就可以大大减少对这些系统的分析工作。从物理上讲,这些自由基的浓度在一个迅速的初始增长后,其消耗与形成的速率就很快趋近,即生成和消耗速率是相等的。

\subsubsection{单分子反应机理}

考虑第三体的作用,产物的生成速率应当等于激活的A分子浓度成一阶反应的速率常数:

\begin{equation}
    \frac{\dd[\mathrm{products}]}{\dd t}=k_\mathrm{uni}[\mathrm{A^*}]
\end{equation}

利用稳态近似,认为激活的A分子浓度不变,可以将A\(^*\)的净生成率表达成为:

\begin{equation}
    {\frac{\mathrm{d}[\mathrm{A}^{*}]}{\mathrm{d}t}}=k_{e}[\mathrm{A}][\mathrm{M}]-k_\mathrm{d e}[\mathrm{A}^{*}][\mathrm{M}]-k_{\mathrm{uni}}[\mathrm{A}^{*}].
\end{equation}

经过层层推导可以得到如下的结论:

\begin{equation}\label{equ:unimol_react}
    -\frac{\dd[\mathrm{A}]}{\dd t}=\frac{\dd[\mathrm{products}]}{\dd t} = k_\mathrm{app}[\mathrm{A}]
\end{equation}其中\(k_\mathrm{app}\)被定义为单分子反应的\textbf{表观速率常数},计算公式为:

\begin{equation}
    k_\mathrm{app}=\frac{k_e[\mathrm{M}]}{(k_\mathrm{de}/k_\mathrm{uni})[\mathrm{M}]+1}
\end{equation}

\subsubsection{链式反应和链式分支反应}

对于总包反应

\begin{equation}
    \mathrm{A_2+B_2}\rightarrow 2\mathrm{AB}
\end{equation}

\textbf{链的激发反应}为:
\begin{equation}
    \mathrm{A_2+M}\overset{k_1}{\longrightarrow} \mathrm{A+A+M}
\end{equation}
\textbf{链的传播反应}为:
\begin{eqnarray}
    \mathrm{A+B_2\overset{k_2}{\longrightarrow}AB + B}\\
    \mathrm{B+A_2\overset{k_3}{\longrightarrow}AB + A}
\end{eqnarray}
\textbf{链的终止反应}为:
\begin{equation}
    \mathrm{A+B+M\overset{k_4}{\longrightarrow}AB+M}
\end{equation}

分析过程太复杂,可以看书100页。几个结论:
\begin{equation}
    [\mathrm{A}]\approx{\frac{[\mathrm{A}_{2}]}{[\mathrm{B}_{2}]^{1/2}}}\left({\frac{k_{1}k_{3}}{k_{2}k_{4}}}\right)^{1/2}
\end{equation}

\begin{equation}
    \frac{\mathrm{d}[{\mathrm{B}}_{2}]}{\mathrm{d}t}\approx-[{\mathrm{A}}_{2}][{\mathrm{B}}_{2}]^{1/2}\left(\frac{k_{1}k_{2}k_{3}}{k_{4}}\right)^{1/2}.
\end{equation}
\begin{enumerate}
    \item 链的激发反应速率越大,链的中断反应速率常数越小,自由基的浓度也越大。
    \item 增大链的传递反应速率常数,会增大\([\mathrm{B_2}]\)的消耗速率。
    \item 链的传递反应速率常数对自由基的浓度影响不大。因为由于其速率常数具有相同的量级且以一个比值的方式出现。
\end{enumerate}

压力足够高时,由于假设的\(4k_2 k_3[B_2]/(k_1 k_4[M]^2)\gg 1\)不再成立,上面的结论也不再可靠。

\textbf{链式分支反应}是指:\textit{消耗一个}自由基而\textit{形成两个}自由基组分的反应。链式分支反应对具有自传播特性的火焰起主导作用,这也是燃烧化学中最基本的特征。

\textbf{化学时间尺度}

\begin{enumerate}
    \item 单分子反应
    我们对单分子反应速率的表达式\ref{equ:unimol_react}进行积分,得到:
    \begin{equation}
        [\mathrm{A}](t)=[\mathrm{A}]_{0}\exp(-k_{\mathrm{app}}t)
    \end{equation}其中[A]\(_0\)为组份A的初始浓度. 如果用RC电路定义特征时间的方法,我们可以得到\textbf{化学时间尺度}的表达式:
    \begin{equation}
        \tau_\mathrm{chem} = 1/k_\mathrm{app}
    \end{equation}

    \item 双分子反应
    对于如下的反应
    \[
        \mathrm{A+B\rightarrow C+D}
    \]
    它的速率表达式为:
    \begin{equation}
        \frac{\dd [\mathrm{A}]}{\dd t}=-k_\mathrm{bimolec}\mathrm{[A][B]}
    \end{equation}
    A和B的浓度变化可以通过化学当量关系来联系, 经过推导之后最后得到结论:
    \begin{equation}
        \tau_{\mathrm{chem}}={\frac{\mathrm{ln}[e+(1-e)\mathrm{([A]_{0}/[B]_{0})}]}{([\mathrm{B}]_{0}-\mathrm{[A]}_{0})k_{\mathrm{bimodec}}}},
    \end{equation}其中\(e=2.718\).

    如果其中反应物浓度要比另一种大得多,比如B很大,那么上面的式子可以简化为:
    \begin{equation}
        \tau_\mathrm{chem}=\frac{1}{[\mathrm{B}]_0 k_\mathrm{bimolec}}
    \end{equation}

    \item 三分子反应
    \[\mathrm{A+B+M\rightarrow C + M}\]
    如果我们认为第三体浓度[M]是一个常数,我们可以得到和双分子反应类似的结论:
    \begin{equation}
        {\frac{\mathrm{d}[\mathrm{A}]}{\mathrm{d}t}}=(-k_{\mathrm{ter}}[\mathrm{M}])[\mathrm{A}][\mathrm{B}].
    \end{equation}

    \begin{equation}
        \tau_{\mathrm{chem}}={\frac{\mathrm{ln}[e+(1-e)([\mathrm{A}]_{0}/[\mathrm{B}]_{0})]}{([\mathrm{B}]_{0}-[\mathrm{A}]_{0})k_{\mathrm{te}}[\mathrm{M}]}},
    \end{equation}


    类似地, 如果B的浓度远远大于A, 那么:
    \begin{equation}
        \tau_{\mathrm{chem}}=\frac{1}{[{\rm B}]_{0}[{\rm M}]k_{\mathrm{ter}}}.
    \end{equation}
\end{enumerate}

\subsubsection{部分平衡}
将\textit{快速反应}视作\textit{平衡态}处理可以简化化学动力学机理,从而无须写出所涉及自由基的速率方程。这种处理方法叫作部分\textbf{平衡近似}。
