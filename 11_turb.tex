\section{湍流概论}
\subsection{概述}
\subsection{湍流的定义}
\subsection{湍流的几何尺度}
\subsubsection{4个几何尺度}
\begin{enumerate}
    \item \(L\)——\textit{流动的特征宽度}或宏观尺度:系统中最大的一个尺度,而且也是最大可能旋涡的上边界。定义平均流速下的雷诺数。
    \item \(\mathcal{l}_0\)——\textit{湍流的积分尺度}或宏观尺度:大旋涡的平均尺寸,这些涡的频率低、波长大。和\(L\)的数量级相同。可以看成是流体中脉动速度不再相关的两点间的距离。
    \item \(\mathcal{l}_\lambda\)——\textit{泰勒微尺度}:这一尺度与平均应变率有关,数量级在\(\mathcal{l}_0\)和\(\mathcal{l}_\mathrm{K}\)之间。它是黏性耗散开始影响漩涡的长度尺度。
    \item \(\mathcal{l}_\mathrm{K}\)——\textit{柯尔莫哥洛夫(Kolmogorov)微尺度}:代表了湍流动能耗散为流体内能的尺度。
\end{enumerate}