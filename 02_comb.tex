\section{燃烧与热化学}
\subsection{概述}
\subsection{热力学参数关系式回顾}
\subsubsection{广延量和强度量}
\textbf{广延量:}取决于物质的数量(质量或物质的量),一般大写;{\textbf{强度量:}单位质量(或物质的量)来表示},数值与物质的量无关。\textit{单位物质的量的在本书中会加上划线},如$\overline{u}$,单位质量的则不加划线,如$u$。

\subsubsection{状态方程}
\begin{eqnarray}
    PV&=&nR_uT\\
    PV&=&mRT\\
    Pv&=&RT\\
    P&=&\rho RT
\end{eqnarray}
$R_u=8315~\mathrm{J/(kmol\cdot K)}$, $R=R_u/\mathrm{MW}$, $\rho=1/v=m/V$.

\subsubsection{状态热方程}
\begin{multicols}{2}
    \tiny
    \begin{eqnarray*}
        u&=&u(T,v)\\
        h&=&h(T, P)
    \end{eqnarray*}
    \begin{eqnarray*}
        \mathrm{d}u&=&\left({\frac{\partial u}{\partial T}}\right)_{v}\mathrm{d}T+\left({\frac{\partial u}{\partial v}}\right)_{T}\mathrm{d}v\\ 
        \mathrm{d}h&=&\left({\frac{\partial h}{\partial T}}\right)_{p}\mathrm{d}T+\left({\frac{\partial h}{\partial P}}\right)_{T}\mathrm{d}P
    \end{eqnarray*}
    \begin{eqnarray*}
        c_{v}&\equiv&\left(\frac{\partial u}{\partial T}\right)_{v}\\
    c_{p}&\equiv&\left({\frac{\partial h}{\partial T}}\right)_{P}
    \end{eqnarray*}
    
    对于理想气体,$(\partial u/\partial v)_T$和$(\partial h/\partial P)_{T}$都为0。所以理想气体的状态热方程为:
    \begin{eqnarray*}
        u(T)-u_{\mathrm{ref}}&=&\int_{T_{\mathrm{ref}}}^{T}c_{v}\,\mathrm{d}T\\ 
        h(T)-h_{\mathrm{ref}}&=&\int_{T_{\mathrm{ref}}}^{T}c_{p}\,\mathrm{d}T.
    \end{eqnarray*}
    Maxwell relations:
    \begin{eqnarray*}
        +\left(\frac{\pp T}{\pp V}\right)_{S} = -\left(\frac{\pp P}{\pp S}\right)_{V} &=& \frac{\pp^2 U}{\pp S \pp V} \\
        +\left(\frac{\pp T}{\pp P}\right)_{S} = +\left(\frac{\pp V}{\pp S}\right)_{P} &=& \frac{\pp^2 H}{\pp S \pp P} \\
        +\left(\frac{\pp S}{\pp V}\right)_{T} = +\left(\frac{\pp P}{\pp T}\right)_{V} &=& -\frac{\pp^2 F}{\pp T \pp V} \\
        -\left(\frac{\pp S}{\pp P}\right)_{T} = +\left(\frac{\pp V}{\pp T}\right)_{P} &=& \frac{\pp^2 G}{\pp T \pp P}
    \end{eqnarray*}

    \begin{equation}
        A = U-TS
    \end{equation}
\end{multicols}


\subsubsection{理想气体混合物}
组份$i$的摩尔分数$\chi_i$:
\begin{equation}
    \chi_{i}\equiv\frac{N_{i}}{N_{1}+N_{2}+\cdots+N_{i}+\cdots}=\frac{N_{i}}{N_{\mathrm{tot}}}
\end{equation}
组份$i$的质量分数$Y_i$:
\begin{equation}
    Y_{i}\equiv\frac{m_{i}}{m_{1}+m_{2}+\cdots+m_{i}+\cdots}=\frac{m_{i}}{m_{\mathrm{tot}}}
\end{equation}
他们之间存在着如下的换算关系:
\begin{equation}
    Y_{i}=\chi_{i}\mathrm{M}\mathrm{W}_{i}/\mathrm{M}\mathrm{W}_{\mathrm{mix}}
\end{equation}
\begin{equation}
    \chi_{i}=Y_{i}\mathrm{MW}_{\mathrm{mix}}/\mathrm{MW}
\end{equation}
对于混合物的摩尔质量:
\begin{equation}
    \mathrm{MW}_\mathrm{mix} = \sum_i \chi_i \mathrm{MW}_i
\end{equation}
\begin{equation}
    \mathrm{MW}_\mathrm{mix} = \frac{1}{\sum_i (Y_i/\mathrm{MW}_i)}
\end{equation}

混合物的强度量可以用各物质的强度量加权计算得到,对于组份的熵,我们有:
\begin{equation}
    s_{i}(T,P_{i})=s_{i}(T,P_{\mathrm{ref}})-R\ln{\frac{P_{i}}{P_{\mathrm{ref}}}}
\end{equation}
\begin{equation}
    \bar{s}_{i}(T,P)=\bar{s}_{i}(T,P_{\mathrm{ref}})-R_{u}\ln{\frac{P_{i}}{P_{\mathrm{ref}}}}\,.
\end{equation}

\subsubsection{蒸发潜热}
aka 蒸发焓,
\begin{equation}
    h_{fg}(T,P)\equiv h_{\mathrm{vapor}}(T,P)-h_{\mathrm{liquid}}(T,P),
\end{equation}

给定温度和压力计算蒸发潜热的方法,Clausius-Claperon方程,
\begin{equation}
    \frac{\mathrm{d}P_{\mathrm{sat}}}{P_{\mathrm{sat}}}=\frac{h_{f g}}{R}\,\frac{\mathrm{d}T_{\mathrm{sat}}}{T_{\mathrm{sat}}^{2}}.
\end{equation}

\subsection{热力学第一定律}
\subsubsection{第一定律——定质量}
\subsubsection{第一定律——控制体}

\subsection{反应物和生成物的混合物}
\subsubsection{化学计量学}
对于碳氢燃料C$_x$H$_y$,
\begin{equation}
    \begin{aligned}
        &C_x\mathrm{H}_y + a(\mathrm{O}_2 + 3.76\mathrm{N}_2)\rightarrow\\
        & x\mathrm{CO}_2 + \frac{y}{2}\mathrm{H}_2\mathrm{O} + 3.76a\mathrm{N}_2
    \end{aligned}
\end{equation}



其中,
$$
a=x+y/4.
$$
\textbf{化学当量的空-燃比}:
\begin{equation}
    (A/F)_{\mathrm{stolc}}=\left(\frac{m_{\mathrm{air}}}{m_{\mathrm{fuel}}}\right)_{\mathrm{stoic}}=\frac{4.76a}{1}\frac{M W_{\mathrm{air}}}{MW_{\mathrm{fuel}}},
\end{equation}
\textbf{当量比}:
\begin{equation}
    \Phi={\frac{(A/F)_{\mathrm{stoic}}}{(A/F)}}={\frac{(F/A)}{(F/A)_{\mathrm{stoic}}}}
\end{equation}
当量空气百分比=100\%/$\Phi$,过量空气百分比=(1-$\Phi$)/$\Phi\times$100\%

\subsubsection{绝对(或标准)焓和生成焓}

绝对焓=标准生成焓+显焓的变化,

\begin{equation}
    \overline{h}_i(T) = \overline{h}_{f,i}^0(T_\mathrm{ref})+\Delta\overline{h}_{s,i}(T),
\end{equation}

\textbf{参考温度:}$T_\mathrm{ref}=25^\circ \mathrm{C}$(298.15 K),\textbf{参考压力:}$P_\mathrm{ref}=$1atm(101 325 Pa)。

对于\textbf{标准生成焓}:元素最自然状态时的生成焓为0,比如氧气,氮气等。

\subsubsection{燃烧焓和热值}
\textbf{燃烧焓}定义为(反应物和产物\textit{都处于标准状态下}):
\begin{equation}
    \Delta h_{R}\equiv q_{c v}=h_{\mathrm{prod}}-h_{\mathrm{reac}},
\end{equation}
\textbf{燃烧热}$\Delta h_c$(也称为热值)为燃烧焓的相反数。
\begin{itemize}
    \item \textbf{高位热值}(HHV):假设所有的产物都凝结成液化水时的燃烧热。
    \item \textbf{地位热值}(LHV):没有水凝结成液态的情况下的燃烧热。
\end{itemize}

\subsection{绝热燃烧温度}

\textbf{定压绝热燃烧温度: }
\begin{equation}
    h_{\mathrm{reac}}(T_{i},P)=h_{\mathrm{prod}}(T_{a d},P).
\end{equation}

\begin{figure}[H]
    \centering
    \includegraphics[width=.23\textwidth]{img/ad_T.png}
\end{figure}

\textbf{定容绝热燃烧温度:}反应前后内能相等,
\begin{equation}
    U_{\mathrm{reac}}(T_{\mathrm{init}},P_{\mathrm{init}})=U_{\mathrm{prod}}(T_{a d},P_{f}),
\end{equation}

写成焓的形式:
\begin{equation}
    \begin{aligned}
        H_{\mathrm{reac}}-H_{\mathrm{prod}}-V(P_{\mathrm{init}}-P_{f})&=0.\\
        H_{\mathrm{reac}}-H_{\mathrm{prod}}-R_{u}(N_{\mathrm{reac}}T_{\mathrm{init}}-N_{\mathrm{prod}}T_{a d})&=0.
    \end{aligned}
\end{equation}

\subsection{化学平衡}
\subsubsection{第二定律的讨论}

单个组份的熵计算公式:
\begin{equation}
    \overline{{{s}}}_{i}=\overline{{{s}}}_{i}^{0}(T_{\mathrm{ref}})+\int_{T_{\mathrm{ef}}}^{T_{f}}\overline{{{c}}}_{p,i}\,\frac{\mathrm{d}T}{T}-R_{u}\,\ln\frac{P_{i}}{P^{0}},
\end{equation}

对于封闭系统,反应自发发生的条件为$\mathrm{d}S\ge 0$。平衡条件为:$(\mathrm{d}S)_{U,V,m}=0$。

\subsubsection{吉布斯函数}

单个组份的吉布斯函数的计算:

\begin{equation}
    \overline{{{g}}}_{i,T}=\overline{{{g}}}_{i,T}^{o}+R_{u}T\ln\left(P_{i}/P^{o}\right)
\end{equation}

对于开口系统,我们采用吉布斯函数,它的定义为 $G\equiv H-TS$。这是第二定律表示为$(\mathrm{d}G)_{T,P,m}\le 0$的形式。在平衡时,开口系统的第二定律可以写作$(\mathrm{d}G)_{T,P,m}=0$。

{
    \scriptsize
    考虑广延量,理想气体的吉布斯方程为:
    \[
        G_{\mathrm{mix}}=\sum N_{i}\overline{{{g}}}_{i,T}=\sum N_{i}\bigl[\bar{g}_{i,T}^{0}+R_{u}T\ln\bigl(P_{i}/P^{0}\bigr)\bigr]
    \]
    对上面的式子取微分,得到平衡条件,可以写作:
    \[
        \begin{aligned}
        &\sum{\mathrm{d}{N}}_{i}\left[\bar{g}_{i,T}^{0}+R_{u}T\ln\left(P_{i}/P^{0}\right)\right]+\\
        &\sum{N}_{i}\mathrm{d}\left[\bar{g}_{i,T}^{0}+R_{u}T\ln\left(P_{i}/P^{0}\right)\right]=0.
        \end{aligned}
    \]
    考虑到总压不变,也就是分压变化的和不变,因此式子中的第二项等于零,它可以被简化为:
    \[\sum{\mathrm{d}{N}}_{i}\left[\bar{g}_{i,T}^{o}+R_{u}T\ln\left(P_{i}/P^{0}\right)\right]=0\]
    对于一个一般的系统,我们将化学反应写作
    \[a\mathbf{A}+b\mathbf{B}+\cdots\leftrightarrow e\mathbf{E}+f\mathbf{F}+\cdots\]
    由于物质的摩尔数变化和化学计量数成正比,因此我们可以将平衡表达式展开写作:
    \[
        \begin{aligned}
            &-a\Bigl[\bar{g}_\mathrm{A,T}^{o}+R_{u}T\ln\bigl(P_{\mathrm{A}}/P^{o}\bigr)\Bigr]\\
            &-b\Bigl[\bar{g}_{\mathrm{B,T}}^{o}+R_{u}T\ln\bigl(P_{\mathrm{B}}/P^{o}\bigr)\Bigr]-\cdots\\
            &+e\Bigl[\overline{{{g}}}_{\scriptscriptstyle\mathrm{E,}T}^{o}+R_{u}T\ln\bigl(P_{\scriptscriptstyle\mathrm{E}}/P^{o}\bigr)\Bigr]\\
            &+f\Bigl[\overline{{{g}}}_{\scriptscriptstyle\mathrm{F,}T}^{o}+R_{u}T\ln\bigl(P_{\scriptscriptstyle\mathrm{F}}/P^{o}\bigr)\Bigr]+\cdots=0.
        \end{aligned}
    \]
    合并整理一下不难得到:
    \[
        \begin{aligned}
            &-\Bigl(e\bar{g}_{\mathrm{E},T}^{o}+f\overline{{{g}}}_{\mathrm{F},T}^{o}+\cdot\cdot\cdot-a\overline{{{g}}}_{\mathrm{A},T}^{o}-b\overline{{{g}}}_{\mathrm{B},T}^{o}-\cdots\Bigr)\\
            &=R_{u}T\ln{\frac{\left(P_{\mathrm{E}}/P^{o}\right)^{e}\cdot\left(P_{\mathrm{F}}/P^{o}\right)^{f}\cdot\mathrm{etc.}}{\left(P_{\mathrm{A}}/P^{o}\right)^{a}\cdot\left(P_{\mathrm{B}}/P^{o}\right)^{b}\cdot\mathrm{etc.}}}
        \end{aligned}\]
}
我们定义\textbf{标准状态吉布斯函数差 $\Delta G_T^0$}为:
\begin{equation}
    \Delta G_T^0 = (e\bar{g}_{\mathrm{E},T}^{o}+f\overline{{{g}}}_{\mathrm{F},T}^{o}+\cdot\cdot\cdot-a\overline{{{g}}}_{\mathrm{A},T}^{o}-b\overline{{{g}}}_{\mathrm{B},T}^{o}-\cdots)
\end{equation}
\textbf{平衡常数 \(K_p\)}为:
\begin{equation}
    K_{p}={\frac{\left(P_{\mathrm{E}}/P^{o}\right)^{e}\cdot\left(P_{\mathrm{F}}/P^{o}\right)^{f}\cdot\mathrm{etc.}}{\left(P_{\mathrm{A}}/P^{o}\right)^{a}\cdot\left(P_{\mathrm{B}}/P^{o}\right)^{b}\cdot\mathrm{etc.}}}.
\end{equation}
这时,定压,定温条件下的化学平衡表达式就可以被写作:
\begin{equation}
    \Delta G_T^0 = -R_u T\ln K_p
\end{equation}
\begin{itemize}
    \item 如果\(\Delta G_T^0\)大于零,那么\(K_p\)小于1,反应向左进行(偏向反应物、几乎不反应)。
    \item 如果\(\Delta G_T^0\)小于零,那么\(K_p\)大于1,反应向右进行(偏向产物,趋于完全反应)。
\end{itemize}

{
    \scriptsize
    如果将\(\Delta G_T^0\)写作:
    \[
        \Delta G_T^0 = \Delta H^0 - T\Delta S^0
    \]
    的形式,平衡常数可以被写作:
    \[
        K_p = \mathrm{e}^{-\Delta H^0/R_u T}\cdot \mathrm{e}^{\Delta S^0/R_u}
    \]
    不难发现,
    \begin{itemize}
        \item 当反应的焓变小于零,反应放热,系统能量降低;
        \item 熵变大于零。
    \end{itemize}都会导致反应偏向于产物,\(K_p>1\)。
}

\subsubsection{复杂系统}

\subsection{燃烧的平衡产物}
\subsubsection{全平衡}
考虑实际的燃烧过程,最大燃烧温度一般发生在略微富燃料当量比的状态(\(\Phi\approx 1.05)\)。
